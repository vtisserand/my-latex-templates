\section{Highlights}

\begin{frame}{Rough Volatility}
    \begin{table}[H]
\centering 
\begin{adjustbox}{width=12cm,center}
\resizebox{\textwidth}{!}{\begin{tabular}{c c c c c}
\hline
\textbf{Model name} &   ${K(t)}$ & \textbf{Domain of ${H}$} & \textbf{Semi-mart.} & \textbf{Markovian} \\
%	[0.5ex] % inserts table
%heading
\hline 
\textit{rough} & $\eta t^{H-1/2}$ & $(0,1/2]$ & \xmark  & \xmark \\
\textit{path-dependent} & $\eta (t+\varepsilon)^{H-1/2}$ & $(-\infty,1/2] $ & \cmark   & \xmark  \\
\textit{one-factor} & $\eta \varepsilon^{H-1/2}e^{-(1/2-H)\varepsilon^{-1} t}$ & $(-\infty,1/2]$ & \cmark   & \cmark  \\
$\textit{two-factor}$ & \makecell{$\eta_1 \varepsilon^{H_1-1/2} e^{-(1/2-H_1)\varepsilon^{-1} t}+$\\ $\eta_2 \varepsilon^{H_2-1/2} e^{-(1/2-H_2)\varepsilon^{-1} t}$} & $(-\infty,1/2]$ & \cmark   & \cmark \\
\hline
\end{tabular}}
\end{adjustbox}
\caption{Different kernels used through the paper, table and names from \cite{abi2024volatility}.}
\label{tab:kernels}
\end{table}
\end{frame}

\begin{frame}{fBM}

    We estimate a Hurst exponent $H \approx 0.14$, consistent with \cite{gatheral2022volatility}.

    A convenient way to get rougher volatility is to change the stochastic driver to a fractional Brownian motion. 
    \begin{block}{Fractional Brownian motion}
    This process $B_t^H$ is a continuous zero-mean Gaussian with covariance function
    \[
    \mathbb{E}[B_t^H B_s^H] = \dfrac{1}{2} \left( \lvert t \rvert ^{2H} + \lvert s \rvert ^{2H} - \lvert t-s \rvert ^{2H}  \right).
    \]
    \end{block}

    This yields models that are neither Markovian nor semimartingales! Is it a good tradeoff?
\end{frame}

\begin{frame}{Solving the optimization problem}

    We've tested different optimizers but choose to go with SLSQP, in particular because it accepts constraints.
    
    \begin{exampleblock}{Initial guess}
        We start the optimization with the set of parameters: \[\Theta = \{a,b,c,H,\eta,\rho,\xi_0,\varepsilon \} = \{0.3, 0.1, 0.0025, 0.14, 0.7, -1.0, 0.3, 1/52\},\]
        although we have empirical proxies for $a$, $b$ and $c$ (convexity, skew and ATM vol level).
    \end{exampleblock}
\end{frame}